\documentclass{article}
\usepackage[utf8]{inputenc}

\title{Domande Teoriche Esame TDS}
\author{Federico Coccarelli - Iris Curioso- Michele Leigheb}
\date{Gennaio 2022}
\usepackage{amsfonts}
\usepackage{amsmath}
\usepackage{mathrsfs}
\usepackage{hyperref}
\usepackage{mathtools}

\usepackage{titlesec}
\usepackage[ margin=1in]{geometry}
\usepackage{graphicx}
\hypersetup{
    colorlinks,
    citecolor=black,
    filecolor=black,
    linkcolor=black,
    urlcolor=black
}
\DeclareUnicodeCharacter{2212}{-}
\DeclareUnicodeCharacter{2200}{\forall}
\DeclareUnicodeCharacter{03B5}{\varepsilon}
\DeclareUnicodeCharacter{2203}{\exists}
\DeclareUnicodeCharacter{03B4}{\delta}
\DeclareUnicodeCharacter{2264}{\le}
\DeclareUnicodeCharacter{2265}{\ge}
\DeclareUnicodeCharacter{2019}{'}


\newcommand{\incfig}[2]{%
    %\def\svgwidth{\columnwidth}
	\includegraphics[scale = #2]{./figures/#1}%{#1}
}

%\sectionfont{\fontsize{13}{15}\selectfont}
%\setlength{\parindent}{
%0em}+


\begin{document}
\large
\maketitle

%\large
\tableofcontents{}


\section{Domande di Teoria}

\subsection{Forma di Kalman di un sistema}
Abbiamo il nostro sistemino
\[ \begin{cases} \overset{\cdot}{x} = Ax+Bu\\y=Cx+Du \end{cases}\]
Prima ricaviamo il sottospazio degli inosservabili:
\[
\mathfrak{I} = ker(O) = ker\begin{pmatrix}C\\CA\\...\\CA^{n-1}\end{pmatrix}  
\]
Poi i raggiungibili:
\[
    \mathfrak{R} = Im(R) = Im\begin{pmatrix}B&AB&...&A^{n-1}B\end{pmatrix}
\]
Trovati questi due calcolo l'intersezione, cioè il sottospazio degli stati inosservabili e raggiungibili,
che chiamo $\chi_1$
\[
  \chi_1 =   \mathfrak{I}\bigcap\mathfrak{R}
\]
Poi calcolo $\chi_2 | \chi_2 \oplus  \chi_1 = \mathfrak{R}$, cioè i raggiungibili osservabili.
Poi calcolo $\chi_3 | \chi_3 \oplus  \chi_1 = \mathfrak{I}$, cioè i irraggiungibili inosservabili.
E alla fine $\chi_4 | \chi_1\oplus \chi_2 \oplus\chi_3 \oplus\chi_4  = \mathbb{R}^n$.
Che credo siano irraggiungibili osservabili.
Da qui mi ricavo la matrice di cambiamento di base:
\[ T^{-1} = \begin{pmatrix}\chi_1&\chi_2&\chi_3&\chi_4 \end{pmatrix}\]
Quindi alla fine mi viene 
\[
    \overset{\sim}{A} = TAT^{-1}\quad \overset{\sim}{B} = TB \quad \overset{\sim}{C} = CT^{-1}
\]


\subsection{Definizione smorzamento e pulsazione naturale}
Nel caso di autovalori complessi, cioè con una $W(s)$ che abbia  al denominatore 
dei poli del tipo $(s-s_0)(s-s_0*)$, con $s_0 = \alpha+j\omega$,
posso introdurre due parametri che caratterizzano il modo in funzione di $\alpha$ e $\omega$.
Si tratta della pulsazione naturale $\omega _n = \sqrt{\alpha^2 + \omega^2}$,
e dello smorzamento $\xi = \sin(\theta) = \frac{-\alpha}{\sqrt{\alpha^2 + \omega^2}} $.

$\omega_n$ rappresenta la pulsazione che avrebbe il sistema se non ci fosse la parte reale

Lo smorzamento rappresenta l'attenuazione a cui è sottoposto il processo oscillatorio del sistema.
In base al suo valore è possibile capire alcune caratteristiche del sistema:
\begin{itemize}
    \item $\xi = \pm 1 \to $ Gli autovalori sono reali e coincidenti, nessun fenomeno oscillatorio.
    \item $\xi = 0 \to$ Gli autovalori sono immaginari puri, si hanno quindi moti oscillatori puri.
    \item $0<\xi<1 \to$ Il modo è pseudoperiodico convergente.
    \item $-1<\xi<0 \to$ Il modo è  pseudoperiodico divergente.
\end{itemize}


\subsection{Problema della realizzazione}
Il problema della realizzazione consiste nell'associare ad una data matrice di funzioni,
una rappresentazione dello stato, cioè di simulare in tempo reale il comportamento del sistema.
Assegnato un legame funzionale, definito dall'integrale di convoluzione con il nucleo $K(t)$ cioè :
\[ Y(t)=\int_{0}^{t}K(t-\tau)u(\tau) \]
il problema ammette soluzione se il nucleo appena descritto coincide con la matrice delle risposte impulsive di un sistema, cioè :
\[ K(t)=Ce^{At}B+D\delta(t) \]
La CNES (condizione necessaria e sufficiente) per la realizzazione è che $K(s)=C(sI-A)^{-1}B+D$ sia una matrice di funzioni razionali \textbf{proprie}.

Ciò significa che se la trasformata di Laplace di $K(t)$ non è una matrice di funzioni razionali proprie,
il legame ingresso-uscita non può essere realizzato mediante un sistema lineare.
Se ho che $K(s)$ è una matrice razionale strettamente propria posso quindi risolvere la realizzazione con delle forme canoniche raggiungibili o osservabili.


\subsection{Relazione tra poli di W(s) e gli autovalori}

Le radici del denominatore di $W(s)$ sono dette poli della funzione, sono un sottoinsieme degli autovalori della matrice dinamica del sistema.
In particolare sono gli autovalori simultaneamente osservabili ed eccitabili.


\subsection{Scomposizione canonica rispetto alla raggiungibilità}
Uno stato $x$ è raggiungibile al tempo t a partire dallo stato $x_0$ se esiste un istante di tempo $t_0 $
ed un ingresso $u$, da $t_0$ a $t$, che porta lo stato $x$ al tempo $t$.

Nei sistemi lineari \textbf{x è RAGGIUNGIBILE} da
$x_0=0 \Longleftrightarrow x \in \mathfrak{R} = \mathfrak{Im}(B\ AB\ ...\ A^{n-1}B)$ .

Se,con $n$ la dimensione di $A$, ho $rg(B\ AB\ ...\ A^{n-1}B)=m<n$ esiste una matrice T non singolare tale che:\\\\
\[
TAT^{-1}= \begin{pmatrix}
A_{11} & A_{12}\\
0 & A_{22}
\end{pmatrix}
,
\hspace{2em}
TB = \begin{pmatrix}
B_1\\
0
\end{pmatrix}
,
\hspace{2em}
CT^{-1} = \begin{pmatrix}
C_1 & C_2
\end{pmatrix}
\]

con $A_{11}$ matrice (mxm), $B_1$ (mxp) e $C_1$(qxm). Inoltre, la terna $(A_{11},B_1,C_1)$ è tutta raggiungibile.

L'insieme degli stati raggiungibili coincide con il sottospazio generato dalle colonne
linearmente indipendenti della matrice $\mathfrak{R}=(B\ AB\ ...\ A^{n-1}B)$, 
detta matrice di raggiungibilità. Se tale matrice ha rango n tutti gli stati sono raggiungibili
ed il sistame stesso è detto completamente raggiungibile.


\subsection{Definire eccitabilità di un modo naturale e dimostrarne la condizione. Come sono legate l'eccitabilità dei modi naturali con la raggiungibilità del sistema?}
Un modo naturale è eccitabile con impulsi in ingresso se la sua legge temporale compare
nella matrice delle risposte impulsive dello stato $ H(t)=e^{A t}B$,
quindi se contribuisce a misurare l'ingresso nell'evoluzione dello stato.

Dimostro che un modo è eccitabile se $v_i'B\neq 0$:
\[ e^{At}B = \sum_{i=1}^{n}e^{\lambda_i t} u_i v_i'B \]
È evidente che se  $v_i'B = 0$ il modo non può comparire nella risposta impulsiva dello stato


\subsection{Legame tra raggiungibilità del sistema e eccitabilità dei modi}

Un sistema non interamente raggiungibile è modellabile come l'interconnessione tra un sistema
completamente raggiungibile e uno completamente no.
In questo caso verrebbe: 
\[
\widetilde{A} = TAT^{-1}= \begin{pmatrix}
A_{11} & A_{12}\\
0 & A_{22}
\end{pmatrix}
,
\hspace{2em}
\widetilde{B} = TB = \begin{pmatrix}
B_1\\
0
\end{pmatrix},\ 
\widetilde{C}= CT^{-1} = \begin{pmatrix}
C_1 & C_2
\end{pmatrix}
\]

Così modellato risulta evidente che tutti i modi eccitabili sono quelli legati alle matrici $A_{11}$ e $B_1$,
cioè quelli nel sottosistema interamente raggiungibile.



\subsection{Definizione di raggiungibilità di uno stato; caratterizzare l'insieme degli stati raggiungibili per sistemi dinamici lineari stazionari; confrontare il caso tempo continuo e tempo discreto}
Uno stato $x_r$ è detto raggiungibile all'istante T da $x_0$ se esistono un $t_0<T$, ed un ingresso sull'intervallo di tempo $[t_0,T)$
che porta l'evoluzione dello stato da $x_0$ a $x_r$.
\[ \exists t_0 < T,\ \left.u\right|_{[t_0,T]}\ :\ e^{A(T-t_0)}x_0+ \int_{t_0}^{t} e^{A(t-\tau)}Bu(\tau) d\tau = x_r \]
Per la rappresentazione lineare stazionaria viene considerato lo stato $x(t_0)=0$. 
L'insieme degli stati raggiungibili all'istante $T$ è denotato con $\mathcal{R}(T)$ cioè :
\[\mathcal{R}(T)=\left\{ x_r : x= \int_{t_0}^{t} e^{A(t-\tau)}Bu(\tau)d\tau\right\}
\quad, \quad
x_r \in \mathcal{R}=Im(B\ AB ..... A^{T-1}B)  \]

Per i sistemi a tempo continuo la raggiungibilità è differenziale e ha equivalenza con la controllabilità,
invece nei sistemi a tempo discreto la raggiungibilità implica la controllabilità ma non viceversa,
e se uno stato non si riesce a raggiungere in $n$ passi allora non si può più raggiungere.


\subsection{Definizione di raggiungibilità ed inosservabilità; caratterizzazione e proprietà degli insiemi di tali stati}
L'osservabilità riguarda il comportamento stato-uscita.
Uno stato è detto inosservabile quando corrisponde ad un'evoluzione libera in uscita identicamente nulla,
cioè $Ce^{At}x_I=0$ dove $x_I=x_a-x_b$ ovvero due stati indistinguibili.
L'insieme degli stati inosservabili è definito come:
$\mathfrak{I} =\left\{ x \in R^n ; Ce^{At}x=0 , \forall t \geq 0 \right\}$ che equivale al 
$\mathfrak{I}=ker \begin{pmatrix}C\\CA\\...\\CA^{n-1}\end{pmatrix}$,
in cui $\begin{pmatrix}C\\CA\\...\\CA^{n-1}\end{pmatrix}$ è la matrice di osservabilità del sistema.

Il sistema viene detto completamente osservabile se il rango di questa matrice è pari alla dimensione del sistema
e di conseguenza tutti i modi naturali sono osservabili.
Uno stato $x_r$ è detto raggiungibile all'istante $T$ da $x_0$ se $\exists\ t_0<T$, ed un ingresso sull'intervallo si tempo $[t_0,T)$
che porta lo stato da $x_0$ a $x$.
Per la rappresentazione lineare stazionaria viene considerato lo stato $x(t_0)=0$. 
L'insieme degli stati raggiungibili è denotato con $\mathcal{R}$, cioè :
\[ \mathcal{R}(T)=\left\{x_r : x=\int_{t_0}^{t}e^{A(t-\tau)}Bu(\tau)d\tau, u[t_0,T]\right\}
\quad \mathcal{R}=
Im(B\ AB\ .....\ A^{n-1}B)\]




\subsection{Osservabilità nei sistemi lineari}
Presi due stati iniziali $x_{0a},x_{0b}$ distinti, e consideratene le uscite $y_a,y_b$, se le due uscite 
sono uguali per ogni t allora i due stati sono \textbf{indistinguibili}.

Per inciso, le due uscite sono uguali per ogni $t$ se,
considerando 

\[ y_{0i} = \Psi(t)x_{0i} + \int_{0}^{t} W(t-\tau) u(\tau) d\tau \]
si verifica la seguente:
\[
    \begin{array}{rcl}
        y_{0a}-y_{0b} & = & \Psi(t)x_{0a} + \int_{0}^{t} W(t-\tau) u(\tau) d\tau - \Psi(t)x_{0b} - \int_{0}^{t} W(t-\tau) u(\tau) d\tau \\
        0 & = & \Psi(t)x_{0a} - \Psi(t)x_{0b} \  \forall\   t>0
    \end{array}    
\]
Chiamato stato inosservabile la differenza tra i due $x_I = x_{0a}-x_{0b}$
\[
    \begin{array}{rcl}
        y_{0a}-y_{0b}  = 0 & \Longleftrightarrow & \Psi(t)(x_{0a} -x_{0b} ) = 0  \\
        \Psi(t)x_I = C e^{At}x_I \equiv 0
    \end{array}    
\]
Dato che se una funzione è identicamente nulla in un intervallo, lo sono anche
le sue derivate, per il criterio di Cayley-Hamilton si verifica:
\[ C A^n x_I = 0 \Longrightarrow C A^{n+1} x_I = 0 \]
e gli stati che soddisfano questa eguaglianza sono inosservabili.
Per trovare questi stati costruisco il sistema lineare omogeneo:
\[\begin{pmatrix}C\\...\\CA^{n-1}\end{pmatrix}x_I = 0\]
che ammette soluzioni diverse da zero se la matrice non ha rango pieno, e quindi gli stati inosservabili,
se ci sono, sono un sottospazio lineare dello spazio di stato.

L'insieme degli stati inosservabili è 
\[\mathfrak{I} = \{ x\in\mathbb{R}^n | Ce^{At}=0\ \forall\ t > 0\}\]



\subsection{L'osservabilità dei sistemi lineari a tempo discreto: definizione e condizioni}
Definita la matrice di osservabilità :
\[O = \begin{pmatrix}C\\...\\CA^{n-1}\end{pmatrix}\]
L' insieme degli stati inosservabili nel tempo discreto è:
\[\mathfrak{I} = \left\{ x\in\mathbb{R}^n | CA^t=0\ \forall\ t > 0\right\} = ker(O)\]

Per Cayley-Hamilton ho che $Cx_I=0,CAx_I=0,...,CA^{n-1}x_I=0 \Longrightarrow CA^kx_i = \forall k>n$
Nel tempo discreto, se uno stato non è osservabile nei primi n passi certamente dopo rimarrà inosservabile.


\subsection{Per un sistema tempo discreto dimostrare che l'insieme degli stati inosservabili è :}
\[ \text{ker}\begin{pmatrix}C\\...\\CA^{n-1}\end{pmatrix}\]
Se considero l'evoluzione di un sistema $y_L(k)=CA^Kx(0)$, affinché lo stato $x_0 = x(0)$
sia inosservabile $y_L(k)=0\ \forall k$.

Considero il caso in cui k=0 :
l'evoluzione è  $y_L(k)=CA^0x_0=CIx_0=Cx_0$, e per definizione ho $C\ x_0=0$ se $x_0 \in Ker(C)$\\
Considero ora il caso k=1: $y_L(k)=CAx_0$ e 
\[
     CAx_0=0 \Longleftrightarrow x(0) \in Ker(CA)
\]
Quindi per il generico passo $k$ gli stati inosservabili appartengono a $Ker(CA^{k})$.
Per Cayley-Hamilton ho che $Cx_I=0,CAx_I=0,...,CA^{n-1}x_I=0 \Longrightarrow CA^kx_i = \forall k>n$ e quindi 
per un sistema lineare l'insieme degli stati inosservabili è dato quindi da tutti gli stati
che non si osservano in nessuno dei passi, quindi: 
\[
    I=Ker \begin{pmatrix}C\\CA\\...\\CA^{n-1}\end{pmatrix}
\]







\subsection{Perché A21 viene zero sempre nella forza canonica osservabile dopo il cambio di base }

Una proprietà degli stati inosservabili è che
\textbf{ il sottospazio $\mathfrak{I}$ è invariante rispetto a $A$},
cioè $x_I \text{inossservabile} \Longrightarrow Ax_I \text{inossservabile}$, il che vale 
per ogni cambiamento di coordinate.

Consideriamo lo stato $\begin{pmatrix}z_1\\0\end{pmatrix}$,
di certo anche $A\begin{pmatrix}z_1\\0\end{pmatrix}=\begin{pmatrix}\overset{\sim }{z_1}\\0\end{pmatrix}$
sarà inosservabile.
Con questo a mente vediamo $\overset{\sim}{A}z_I$:
\[
    \overset{\sim}{A}z_I=\begin{pmatrix}\overset{\sim }{A_{11}}&\overset{\sim }{A_{12}}\\
        \overset{\sim }{A_{21}}&\overset{\sim }{A_{22}}\end{pmatrix}\begin{pmatrix}z_1\\0\end{pmatrix}=
    \begin{pmatrix}\overset{\sim }{z_1}\\0\end{pmatrix} \Longrightarrow
    \begin{array}{rcl}
    \overset{\sim }{A_{11}}z_{1} &= \overset{\sim }{z_1}&\\
    \overset{\sim }{A_{21}}z_{1} &= 0 &\to  \overset{\sim }{A_{21}}=0
    \end{array}
\]






\subsection{I modi naturali dei sistemi lineari a tempo continuo: definizione e parametri caratteristici}

I modi naturali sono le evoluzioni nello spazio di stato attraverso le quali è possibile esprimere l'evoluzione libera.
In base agli autovalori che vi sono associati si hanno due casi:\\
\begin{itemize}
    \item I modi naturali \textbf{aperiodici} in presenza di autovalori \textbf{reali e distinti}
    Il parametro caratteristico è $\lambda$, che determina l'andamento temporale.
    \begin{itemize}
        \item   Se $\lambda<0$ il modo è convergente
        \item   Se $\lambda>0$ si ha un andamento divergente
        \item   Se $\lambda=0$ si ha un andamento costante.
    \end{itemize}
    \item I modi naturali \textbf{pseudoperiodici}, associati a coppie di autovalori \textbf{complessi coniugati}
    i cui parametri caratteristici sono :
    \begin{itemize}
        \item $\alpha$ (ampiezza): se $\alpha>0$ il modo è una spirale divergente,
        se $\alpha<0$ il modo è una spirale convergente
        e se $\alpha=0$ il modo rimane costante sulla traiettorie di un ellisse.
    \item $\xi$ (smorzamento): al suo aumento  corrisponde un maggior inviluppo della fase oscillatoria.
    \item $\omega_n$ (pulsazione naturale) rappresenta la pulsazione del modo quando lo smorzamento è nullo.
    \end{itemize}
\end{itemize}
\[ \omega_n=\sqrt{\alpha^2+\omega^2}=\frac{\sqrt{1+\xi^2}}{\omega}\quad , \quad \xi=sen(\theta)=\frac{-\alpha}{\omega_n} \]



\subsection{Modi naturali per sistemi a tempo discreto}
I modi naturali per il caso discreto,così come per i sitemi tempo continuo, sono le evoluzioni nello spazio di stato attraverso le quali è possibile esprimere l'evoluzione libera. In base agli autovalori che vi sono associati si hanno tre casi:\\
\begin{itemize}
    \item Modi naturali aperiodici associati ad autovalori reali positivi.
    Il parametro caratteristico è $\lambda$ che determina l'andamento temporale.
    \begin{itemize}
        \item Se $|\lambda|=1$ si ha un modo costante
        \item Se $|\lambda|<1$ si ha un modo convergente a 0
        \item Se $|\lambda|>1$ si ha un modo divergente.
    \end{itemize}
    \item Modi naturali alternanti in presenza di autovalori reali negativi,
    anche quì il parametro caratteristico è $\lambda$ e:
    \begin{itemize}
        \item Se $\lambda=-1$ il modo oscilla tra -1 e 1
        \item Se $\lambda>-1$ converge a 0 oscillando
        \item Se $\lambda<-1$ diverge oscillando.
    \end{itemize}
    \item modi naturali pseudoperiodici associati a coppie di autovalori complessi coniugati: in questo caso
    scegliamo la rappresentazione polare $\alpha + i \omega = \rho e^{i\theta}$ con
    $\rho = \sqrt{\alpha^2+\omega^2}, \theta = arctan\left(\frac{\omega}{\alpha}\right)$, quindi
    i parametri caratteristici sono : 
    \begin{itemize}
        \item $\rho$ (ampiezza) , se $\rho>1$ il modo è una spirale divergente,
        se $\rho<1$ il modo è una spirale convergente e se $\rho=1$
        il modo rimane costante sulla traiettoria di un' ellisse.
        \item $\theta$ (fase) rappresenta la fase del modo.
    \end{itemize}
\end{itemize}



\subsection{Può un sistema lineare essere caratterizzato da modi naturali aventi la stessa legge di moto?}
Si, se la matrice A ha autovalori coincidenti ma è regolare $(M_G=1)$,
si hanno in corrispondenza di ogni autovalore con molteplicità algebrica $>1$,
altrettanti modi naturali con leggi temporali coincidenti.

In corrispondenza di ogni autovalore reale è possibile calcolare tanti autovettori quanto è la
$M_A$ dello stesso e in corrispondenza di autovalori complessi e coniugati
è possibile calcolare tanti autospazi di dimensione 2 quanta è la $M_A$
della coppia di autovalori.



\subsection{\boldmath Dato un sistema a tempo continuo $(A, B, C, D)$ con $D = 0$, $\Lambda$
l'insieme degli
autovalori, $\Lambda_e$ e $\Lambda_o$ gli insiemi degli autovalori associati ai modi eccitabili e
osservabili. Dimostrare che $\Lambda_e  \bigcap\Lambda_o = \emptyset \Longrightarrow W(t) = 0$}

La risposta impulsiva dipende dalla matrice delle risposte impulsive
\(\displaystyle W(t)=Ce^{At}B=\sum_{i=1}^{n}C u_i v_i' B\).

In questa rappresentazione $C u_i$ è non nulla solo per i modi osservabili,
e $v_i'B$ è non nulla solo per i modi eccitabili,
quindi se un autovalore non è osservabile $(Cu_i=0)$ o non è eccitabile $(v_iB=0)$
non compare in $W(t)$.
Se nessun autovalore compare in $W(t)$, quest'ultima è nulla e allora la risposta impulsiva è nulla.


\subsection{Definizione di guadagno e interpretazione fisica}
Il guadagno è il fattore costante che compare nella forma di Bode della funzione di trasferimento,
esso è definito per ogni elemento ed è pari al valore che assume la $W(s)$
in $s=0$ dopo aver eliminato l'eventuale eccesso di poli in zero.
Il guadagno della funzione di trasferimento di un sistema ha,
nel caso in cui esista la risposta a regime permanente,
il significato fisico di valore di regime della risposta al gradino unitario,
cioè $W(s)|_{s=0}$.



\subsection{Il ruolo dell'osservabilità nella ricostruzione del comportamento interno}
La proprietà di osservabilità svolge in questo problema il
ruolo della proprietà di raggiungibilità nel problema
di assegnazione degli autovalori con retroazione dallo stato.


\subsection{W(t), W(s) e W($j\omega$): cosa rappresentano e qual è il loro significato fisico.}
-W(t) è la matrice delle risposte impulsive in uscita e descrive il comportamento forzato ingresso-uscita. Permette di calcolare la risposta forzata ad un ingresso u(t) andando ad eseguire l'integrale di convoluzione $Y_F(t)=\int_{t_0}^{t} W(t-\tau)u(\tau)d\tau $\\
-W(s) è la trasformata di Laplace di W(t) ed è detta funzione di trasferimento, è un modello forzato di un sistema dinamico lineare stazionario. Conoscendo W(s) si può calcolare una qualsiasi risposta forzata ad un dato ingresso.\\
-W($j\omega$) è la risposta armonica e descrive il comportamento in  frequenza di un sistema e allo stesso tempo consente di dare alla funzione di trasferimento un'interpretazione fisica. Il modulo e la fase della risposta armonica, per una data pulsazione, rendono conto del comportamento a regime permanente a quella pulsazione.


\subsection{Scrivere un sistema (cioè scrivere le matrici $A, B, C$)
in forma canonica osservabile con $n=3, p=q=1$.
Determinare un cambio di base per metterlo in forma canonica raggiungibile}

Data un sistema con funzione di trasferimento 
\[ 
    W(s) = \frac{b_{n-1}s^{n-1}+..+b_0}{a_ns^n+...+a_0}
    \quad\text{che in questo caso è }\quad W(s)=\frac{s^2b_2+sb_1+sb_0}{s^3+a_2s^2+a_1s+a_0}
\]
La forma canonica osservabile sarà:
\[
  A =   \begin{pmatrix}
    0 & 0 & -a_0\\
    1 & 0 & -a_1\\
    0 & 1 & -a_2
    \end{pmatrix},
  B = \begin{pmatrix} b_0\\b_1\\b_2 \end{pmatrix},
  C =  \begin{pmatrix} 0 & 0 &1 \end{pmatrix}
\]
Per passare alla forma canonica raggiungibile lo farei come studio della raggiungibilità.
\small Ma mica sono così sicuro vabe.
\large
Nel senso che preso il sistema facico la matrice $R = (B AB A^2B)$, $\mathfrak{R}=Im(R)$,
$T^{-1} = (\mathfrak{R} .completamento)$ etc. etc.



\subsection{Proprietà di semigruppo}
Le matrici $\Phi(t),H(t),\Psi(t),W(t)$ ddevono soddisfare le seguenti proprietà:

\begin{enumerate}
    \item Proprietà di semigruppo : \[ \Phi(0)=1,\Phi(t-t_0) = \Phi(t-t_1)\Phi(t_1-t_0) \forall\ t_-\le t_1\le t\]
    \item Fattorizzazione sulle matrici $H,\Psi,W$:
    \begin{align*}
        H(t-\tau) = \Phi(t-t_1)H(t_1-\tau) &,& \Psi(t-t_0)=\Psi(t-t_1)\Phi(t_1-t_0)\\
        W(t-\tau) = \Psi(t-t_1)H(t_1-\tau) &
    \end{align*}
\end{enumerate}



\subsection{Osservatore Asintotico}
Allora io ho un sistema $S$ di cui voglio sapere lo stato ma ho solo l'uscita. Devo inventarmi un modo per stimare lo stato.

Credo un nuovo sistema 
\[
    S_2 : \begin{cases} \overset{\cdot}{z} = Az+Bu\\
        y = Cz 
    \end{cases}    
\]
Dato che io sto cercando un sistema che mi emuli il primo, ha senso chiamare "errore" la differenza tra lo stato di uno e dell'altro:
\[
e = z-x \Longleftrightarrow \overset{\cdot}{e}  = \overset{\cdot}{z}-\overset{\cdot}{x} = A(z-x) = Ae  
\]
Io voglio ovviamente che questo errore vada a 0, infatti questo vorrebbe dire che il comportamento dei due sistemi è uguale:
\[
\lim_{t\to\infty}||e(t)|| = 0 \Longleftrightarrow \lim_{t\to\infty}||z(t)-x(t)|| = 0    
\]

Per risolvere il problema includo l'errore nello stato del secondo sistema:


\[e_y = y-y_z\]
\incfig{oss_asint.jpeg}{0.7}

Così che l'equazione di stato viene:
\[
O : \begin{cases}
    \overset{\cdot}{z}= Az +Bu+K(Cx-Cz) = (A-KC)z+Bu+KCx   \\
    y_z = z
    \end{cases}
\]




\section{Passaggi di Rappresentazione}

\subsection{Un sistema dinamico può esser rappresentato tramite un modello implicito o esplicito. \\
(a) Illustrare i diversi vantaggi e svantaggi delle due rappresentazioni; \\
(b) Per sistemi a tempo discreto lineari e stazionari, indicare i passaggi matematici che permettono
di passare da una rappresentazione all'altra e viceversa.}

Un sistema implicito ha una struttura del tipo:
$$
\begin{cases}
    \dot{x}(t) = Ax(t)+Bu(t)\\
    y(t) = Cx(t)+Du(t)
\end{cases}
$$

Nel tempo continuo permette una simulazione in tempo reale del modello del sistema, è più completo in quanto permette di condurre l'analisi modale con tutti gli autovalori e di analizzare le proprietà del sistema come eccitabilità e osservabilità , ma non fornisce le espressioni dello stato e dell'uscita.\\
Il modello esplicito ha una struttura del tipo:
\[
\begin{cases}
    X(t) = \Phi(t-t_0)X(t_0)+\int_{t_0}^{t}H(t-\tau)u(\tau)d\tau\\
    Y(t) = \Psi(t-t_0)X(t_0)+\int_{t_0}^{t}W(t-\tau)u(\tau)d\tau
\end{cases}
\]
e permette di analizzare il comportamento complessivo forzato e libero.
Tuttavia nella rappresentazione esplicita compaiono solo gli autovalori eccitabili o osservabili.
Avendo la forma esplicita 
\[     x(t) = \Phi(t-t_0)x(t_0)+\int_{t_0}^{t} H(t-\tau)u(\tau)d\tau \]
Arrivo a quella implicita sapendo che 
\[ A = \lim_{t\to 0}\frac{\partial }{\partial t}\Phi(t) \quad,\quad B = H(0) \]


\subsection{Dimostrazione passaggio da forma esplicita a implicita}

\begin{align*}
    x(t)  = & \Phi(t-t_0)x(t_0)+\int_{t_0}^{t}H(t-\tau)u(\tau)d\tau \\
    \lim_{t_0\to t}x(t)  = & \lim_{t_0\to t} \Phi(t-t_0)x(t_0)+\int_{t_0}^{t}H(t-\tau)u(\tau)d\tau \\
    \lim_{t_0\to t}\overset{\cdot}{x}(t)  = &\lim_{t_0\to t} \frac{d}{dt}\Phi(\underbrace{t-t_0}_{\to 0})x(\underbrace{t_0}_{\to t})
        +\int_{t_0}^{t}\frac{d}{dt}H(t-\tau)u(\tau)d\tau\\
        \overset{\cdot}{x}  = & \left.\frac{d}{dt}\Phi(t)\right|_{0}x(t)+H(0)u(t)
\end{align*}

\[ A = \left.\frac{d}{dt}\Phi(t)\right|_{0} \quad,\quad B = \left.H(t)\right|_{0} \]


\subsection{Dimostrazione passaggio da forma implicita a esplicita}

Partiamo dal sistema solito 
\begin{equation}\label{sistImpl}
    S = \begin{cases} \overset{\cdot}{x} = Ax+Bu\\ y = Cx \end{cases}
\end{equation}

$\overset{\cdot}{x} = Ax+Bu$ è un'equazione differenziale lineare non omogenea
$\overset{\cdot}{x} = Ax$ è invece omogenea, e ha come soluzione generale $x(t) = ke^{At}$,
in $t_0$ ha soluzione 
\[ x(t_0) = ke^{A(t_0-t)} \to k = e^{A(t-t_0)}x_0 \]

Per $Bu(t)$ vedo la soluzione particolare in un istante $t$ qualsiasi e con tutti i contributi da $t_0$ a $t$, cioè
\[ \int_{t_0}^{t}e^{A(t-\tau)}Bu(\tau)d\tau \]

Unendo le due mi viene quindi 
\begin{equation}\label{rappEspl}
    x(t) = e^{A(t-t_0)}x(t_0)+\int_{t_0}^{t}e^{A(t-\tau)}Bu(\tau)d\tau
\end{equation}

Questo è valido anche se $A$ è una matrice, verifichiamo che $e^{At}$ si comporta come nel caso
scalare $e^{at}$:
\[
\frac{d}{dt}e^{At} =     \frac{d}{dt}\sum_{k=0}^{\infty}\frac{A^kt^k}{k!} =
 A\sum_{h=0}^{\infty}\frac{A^ht^h}{h!} = Ae^{At}
\]

Ora verifichiamo pure che (\ref{rappEspl}) è soluzione di  (\ref{sistImpl}): 
\begin{align*}
    \overset{\cdot}{x} &=& Ae^{A(t-t_0)}x(t_0)+e^{(t-t)}B u(t) 1 - 0 + \int_{t_0}^{t}e^{A(t-\tau)}Bu(\tau)d\tau\\
    \overset{\cdot}{x} &=& A\underbrace{e^{A(t-t_0)}x(t_0)+ \int_{t_0}^{t}e^{A(t-\tau)}Bu(\tau)d\tau}_{x(t)}+Bu(t)
\end{align*}
e daje è verificato.


Ora verifichiamo l'uscita $y(t)$:
\[
y(t) =Cx(t)+Du(t) \quad \to\quad Ce^{A(t-t_0)}x(t_0)+    \int_{t_0}^{t}e^{A(t-\tau)}Bu(\tau)d\tau+Du(t)
\]
Usando l'impulso $\delta(t)$ utilizzo il teorema per cui
\[
\int_{a}^{b}f(t)\delta(t-\tau)d\tau = f(\tau) \quad\Longrightarrow\quad
Du(t) =   \int_{t_0}^{t} D\delta(t-\tau)u(\tau)d\tau
\]
che mi fa venire alla fine 
\[
    y(t) = Ce^{A(t-t_0)x_0}+\int_{t_0}^{t} \left(Ce^{A(t-\tau)}B+D\delta(t-\tau)\right)u(t)d\tau
\]


\subsection{Sistema a tempo discreto equivalente ad un dato sistema a tempo continuo}
Dato il sistema $S_{tc}$ con ingresso $u(t)$ e uscita $y(t)$, il mio obiettivo è poterlo trattare come se fosse a tempo discreto, cioè in maniera tale
che possa accettare un ingresso discreto $u_d(t)$ e che produca un'uscita $y_d(t)$.
Lo schema sarà quindi:

\incfig{ZOH.jpeg}{0.4}

In questo caso il problema è che il sistema di cui ho già il modello si aspetta un ingresso continuo, quindi dovrò usare lo Zero Order Holder per
farglielo arrivare.

Inoltre anche l'uscita del sistema sarà continua, e quindi per ottenerla disxcreta dovrò usare un campionatore.

\incfig{discreto.jpeg}{0.4}

Siamo ora interessati a calcolare le evoluzioni negli istanti, detti di campionamento, corrispondenti alla variazione dell'ingresso.
Il problema del calcolo del sistema a tempo discreto equivalente è detto anche problema della discretizzazione. Dato il sistema a tempo continuo:
\[
\begin{cases}
    \dot{x} = Ax+Bu\\
    y = Cx+Du
\end{cases}
\]

Lo stato, in forma esplicità, sarà quindi : \(\displaystyle x(t) = e^{A(t-t_0)}x(t_0) + \int_{t_0}^{t} e^{A(t-\tau)}Bu(\tau)d\tau\)
Per studiarlo in forma discreta, detto $T_c$ il tempo di campionamento, pongo $t_0$ come un generico istante $kT_c$, e $t$ come l'istante successivo
$(k+1)T_c$. Così mi viene:
\begin{align*} 
    x((k+1)T_c) &=& e^{AT_c}x(kT_c) &+ \int_{kT_c}^{(k+1)T_c} e ^{A((k+1)T_c-\tau)}B u(\tau) d\tau \\
    x((k+1)T_c) &=& e^{AT_c}x(kT_c) &+ \int_{kT_c}^{(k+1)T_c} e ^{A((k+1)T_c-\tau)}B  d\tau \ u(kT_c)
\end{align*}
ponendo quindi le rappresentazioni 

\(\displaystyle A_d=e^{AT_c}, B_d=\int_{kT_c}^{(k+1)T_c} e^{A((k+1)T_c-\tau)}B d\tau , C_d = C,  D_d=D\)

Il sistema a tempo discreto equivalente, detto $T_c$ il tempo di campionamento è:

\[
\begin{cases}
    x(k+1)=A_d x(k)+B_d u(k)\\
    y(k)=C_d (k) + D_d(k)
\end{cases}
\]


\subsection{Come si passa dalla rappresentazione implicita a quella esplicita nel tempo discreto}
Partiamo dalla rappresentazione $x(t+1) = Ax(t)+Bu(t)$ con $t_0,x(t_0)$.
Noto $u(t_0)$ devo arrivare a: 
\begin{equation}\label{eplsTdiscreto}
    x(t) = \Phi(t-t_0)x(t_0)+\sum_{\tau=t_0}^{t-1}H(t-\tau)u(\tau)
\end{equation}
Provando a calcolare $x(t_0+1),x(t_0+2)$ etc arrivo alla forma
$x(t_0+k) = A^kx(t_0)+\sum_{\tau = 0}^{k-1}A^{k-\tau-1}B u(t_0+\tau)$,
così ponendo $t=k+t_0$ (ossia $k=t-t_0$) verrà
\[
x(t) =A^{t-t_0}x(t_0) +\sum_{\tau=0}^{t-t_0-1}A^{t-t_0-\tau-1}B u(t_0+\tau)   
\]
di cui la prima parte è già simile a (\ref{eplsTdiscreto}).

Ora poniamo $T = \tau-t_0$ e poi $\tau=T$: così mi viene:
\[
    x(t) =A^{t-t_0}x(t_0)+\sum_{\tau=t_0}^{t-1}A^{t-\tau-1}Bu(\tau)
\]
Che è praticamente (\ref{eplsTdiscreto}). A questo punto posso concludere ponendo:
\begin{align*}
\Phi(t-t_0)&=A^{t-t_0} &,& H(t-\tau) = A^{t-\tau-1}B \Longrightarrow\\
\Phi(t) &= A^t &,&H(t)= A^{t-1}B
\end{align*}


\subsection{Come si passa dalla rappresentazione esplicita a quella implicita nel tempo discreto}

Partendo da
\[
    x(t) = \Phi(t-t_0)x(t_0)+\sum_{t=t_0}^{t-1}H(t-\tau)u(\tau)
\]
voglio arrivare a $x(t+1) = Ax(t)+Bu(t)$: 

faccio le sostituzioni $t\to t_0$,$t+1\to t$ (cioè prendo $t_0$ come istante iniziale e $t$ come quello dopo ),
così mi viene:
\begin{align*}
    x(t+1) &=& \Phi(t+1-t)x(t)+\sum_{\tau=t_0}^{t+1-1}H(t+1-\tau)u(\tau)\\
    x(t+1) &=& \Phi(1)x(t)+\sum_{\tau=t}^{t}H(t+1-\tau)u(\tau)\\
    x(t+1) &=& \Phi(1)x(t) +H(1)u(t)\\
\end{align*}
Che ha una forma uguale a
\[\Phi(1) =A \quad,\quad H(1)=B\]



\subsection{Come si passa da rappresentazione a tempo continuo a rappresentazione con Laplace}
Se ho il sistema nella solita forma
\[\begin{cases}\overset{\cdot}{x} = Ax+Bu\\y = Cx+Du \end{cases}\]
le matrici in Laplace sono:
\begin{align*}
     \Phi(s) = (sI-A)^{-1}&,&  H(s) = \Phi(s)B,\\
     \Psi(s) = C \Phi(s)&,& W(s)=C\Phi(s)B+D 
\end{align*}


\subsection{Come si passa da rappresentazione a tempo discreto a rappresentazione con la trasformata Zeta}

Dato il mio bel sistema in forma
\[
    x(t) = \Phi(t-t_0)x(t_0)+\sum_{t=t_0}^{t-1}H(t-\tau)u(\tau)
\]
applico la trasformata
\[
    \underset{t\in\mathbb{Z}}{f(t)}\quad\to\quad
    \underset{z\in\mathbb{C}}{F(z)=\sum_{t=0}^{\infty}f(t)z^{-t}}
\]
Per comodità di calcoli tengo a mente la trasformata di $f(t+1)$
\begin{align*}
    Z[f(t+1)] &=&\sum_{t=0}^{\infty}f(t+1)z^{-t}\\
    &=&\sum_{\tau=1}^{\infty}f(\tau)z^{-(\tau-1)}\\
    &=&z\sum_{\tau}^{\infty}f(\tau)z^{-t}\\
    &=&z\left( \underbrace{\sum_{\tau=1}^{\infty}f(\tau) z^{-t}+f(0)}_{Z[f(t)]}-f(0) \right)\\
    &=&zF(z)-zf(0)
\end{align*}
Allora dato che $F(x(k+1)) = zX(z)-zx(0)$:
\begin{align*}
    zX(z)-zx(0) &=& Ax(z)+Bu(z)\\
    zX(z)-Ax(z) &=& zx(0)+Bu(z)\\
    (zI-A)X(z) &=& zx(0)+Bu(z)\\
    X(z) &=& \underbrace{(zI-A)^{-1}zx(0)}_{Z[\Phi(t)x_0]}+
    \underbrace{(zI-A)^{-1}BU(z)}_{Z[\sum_{t=t_0}^{t-1}H(t-\tau)u(\tau)]}
\end{align*}


\section{Stabilità}

\subsection{Studiare la stabilità di un sistema non lineare}
Prendiamo il sistema 
\[
\begin{cases}
    \overset{\cdot}{x} = f(x(t),u(t))\\
    y(t) = h(x(t),u(t))
\end{cases}    
\]
con $x\in\mathbb{R}^n,u\in\mathbb{R}^p,y\in\mathbb{R}^q,f:\mathbb{R}^nx\mathbb{R}^p\to\mathbb{R}^n,h:\mathbb{R}^nx\mathbb{R}^p\to\mathbb{R}^q$

Dato che la permanenza in unos stato è determinata dall'assenza di variazioni, un punto di equilibrio
è un punto $x_e$ tale che $f(x_e,u) = 0$ per qualche u.
E basta, le definizioni di stabilità sono le solite per i punti di equilibrio.


\subsection{Lyapunov di sistemi non lineari a tempo discreto}
Definita la \textit{funzione di Lyapunov} $V(x)$ introduco la  funzione 
\textit{differenza prima} $\Delta V(x) = V(x(t+1))-V(x(t))$ e il punto d'equilibrio è:
$x_e = f(x_e)$
\paragraph{Teorema $3.8.1$:} Lo stato $x_e$ di equilibrio è localmente stabile asintoticamente se esiste una
funzione $V(x)$ definita positiva in un intorno $I_{x_e}$ tale che la syua differenza prima lungo il moto 
\[ \Delta V(x) = V(x(t+1))-V(x(t)) = V(f(x(t)))-V(x(t)) \] sia \textbf{definita negativa}.

\paragraph{Teorema $3.8.2$:}Lo stato $x_e$ di equilibrio è localmente stabile semplicemente se $\exists V(x)$ definita
positiva in $I_{x_e}$, tale che la sua differenza prima lungo il moto  $V(x) = V(x(t+1))-V(x(t))$ sia 
\textbf{semi-definita negativa}



\subsection{Enunciato e dimostrazione teorema di Nyquist}
Condizione necessaria e sufficiente per la stabilità asintotica di un sistema a retroazione unitaria e che il numero di 
giri che il vettore rappresentativo di $1+F(j\omega)$ compie attorno allo 0 (ossia i giri che fa $F(j\omega)$ attorno a -1),
sia uguale al numero di autovalori a parte reale positiva del sistema a anello aperto.

\subsubsection{Dimostrazione:}
Dato il sistema 
\[S = \begin{cases} \overset{\cdot}{x} = Ax+Bu\\ y = Cx \end{cases}\]
e introducendo una controreazione unitaria all'uscita $u = -y +v$ viene:
\[S_{CH} = \begin{cases} \overset{\cdot}{x} = (A-BC)x+Bv\\ y = Cx \end{cases}\]
Il sistema $S$ lo chiamiamo a anello aperto e $S_{CH}$ a anello chiuso.

Ora poniamo $d_{AP}(s) = |sI-A|$ (il denominatore di $F(s)$) e $d_{CH}(s) = |sI-A+BC|$ il denominatore
(polinomio caratteristico)
della funzione di trasferimento a anello chiuso.

Come sappiamo il sisterma complessivo è stabile se $d_{CH}$ non ha radici a parte reale positiva,
inoltre è dimostrabile che vale la:
\begin{equation}\label{Rapporto}
  \frac{d_{CH}(s)}{d_{AP}(s)} =   |I +C(sI - A)^{-1}B| = |I+F(s)|
\end{equation}
infatti a partire da 
\begin{align*}
    |sI-A+BC| &=& |sI-A||I +(sI - A)^{-1}BC|\\
                &=&|sI-A||I +C(sI - A)^{-1}B|
\end{align*}
mi viene

\[
    \frac{d_{CH}(s)}{d_{AP}(s)}    = \frac{|sI-A+BC|}{|sI-A|} = \frac{|sI-A||I +C(sI - A)^{-1}B|}{|sI-A|}
\]
(che dimostra la (\ref{Rapporto})).

Ora consideriamo un polinomio generico $p(s)$ di grado $n$ con le matrici $p_{1,2,..}$.
Ponendo $s = j\omega$ viene 
\[ 
    p(j\omega) = {k(j\omega-p_1)...(j\omega - p_n)}
\]
Inizialmente studio il caso in cui non ci siano radici nulle:
Ogni fattore $j\omega - p_i$ è rappresentabile nel piano complesso come un vettore tra $p_i$ e $j\omega$.

Ora consideriamo due poli, uno con parte reale positiva $p_i$ e uno negativa $p_j$  con i vettori associati.

\incfig{pipj.jpeg}{0.5}

Facendo variare $\omega$ da $-\infty$ a $+\infty$ gli estremi dei vettori variano posizionandosi
sull'asse immaginario sempre più su. Quindi si può dire che per $\omega=-\infty$ i vettori sono entrambi verticali verso il basso,
e per $\omega=\infty$ verso l'alto.

Da questa consapevolezza cosa posso ricavare? Che per i poli positivi passare da $\omega=-\infty$ a $\omega=\infty$
significa far fare al vettore un giro di $-\pi$, mentre per quelli negativi un giro di $\pi$.

Ora definiamo come $\Delta_{\varphi}(p)$ la variazione di fase dell'intero polinomio
quando $\omega$ varia da $-\infty$ a $\infty$: indicando con $n_p$ il numero di radici a parte reale positiva di $p(s)$
è verificata la:
\begin{equation}\label{varphi}
    \Delta_{\varphi}(p) = (n-n_p)\pi-n_p\pi=n\pi-2\pi n_p
\end{equation}

Ora riscrivendo la (\ref{Rapporto}) calcolando in $s = j\omega$ e calcolando le variazioni di fase,
viene:
\[
\Delta_{\varphi}\left(\frac{d_{CH}(j\omega)}{d_{AP}(j\omega)} \right)= \Delta_{\varphi}\left(|I+F(j\omega)|\right)
\]
Sapendo le proprietà del prodotto tra fasi questa la riscriviamo come
\[
    \Delta_{\varphi}\left(\frac{d_{CH}(j\omega)}{d_{AP}(j\omega)} \right)=
    \Delta_{\varphi}(d_{CH}(j\omega))-\Delta_{\varphi}(d_{AP}(j\omega))
\]
Ora, applicando la (\ref{varphi}) a entrambi i polinomi, indicando $Z_p$ il numero di a radici parte reale positiva
di $d_{CH}(j\omega)$ e con $P_p$ quello di $d_{AP}(j\omega)$ viene
\[
    \Delta_{\varphi}\left(\frac{d_{CH}(j\omega)}{d_{AP}(j\omega)} \right)=
    (n\pi-2\pi Z_p)-(n\pi-2\pi P_p)=2\pi(P_p-Z_p )
\]
Quindi a partire dal diagramma polare di $|1+F(s)|$ è possibile determinare quanti giri 
fa il vettore della funzone attorno allo zero. Indicando questo numer con $N$,
(positivo se le rotazioni sono antirorarie e negativo altrimenti) mi viene:
\[
\Delta_{\varphi}(1+F(j\omega))=2\pi N    
\]
e quindi la (\ref{varphi}) si riscrive come 
\[
    2\pi(P_p-Z_p )=2\pi N \quad \Longrightarrow \quad P_p-Z_p = N
\]
Ovviamente il sistema è asintoticamente stabile solo se $Z_p=0$ e quindi 
\[N= P_p\quad\quad \text{è quello che volevo dimostrare}\]


\subsection{Definizione punto di equilibrio}
Consideriamo il sistema deifnito dall'equazione di stato: $ \overset{\cdot}{x} = f(x(t),u(t)) $, un punto di 
equilibrio $x_e$ del sistema è definito dal fatto che $f(x_e,u_e) = 0$ per qualche $u_e$.



Ci limitiamo allo studio del caso in cui $u_e = 0$.

Quindi un sistema ammette punti di equilibrio se esistono soluzioni al sistema :$f(x_e,0) = 0$.

\incfig{x_e.jpeg}{0.5}

Consideriamo lo studio di un punto $x_o$ vicino a $x_e$, ci sono tre possibilità:
\begin{enumerate}
    \item Il sistema rimane in un intorno di $x_e$ ( stabilità semplice)
    \item Il sistema tende a $x_e$ (stabilità asintotica)
    \item Il sistema diverge (instabilità)
\end{enumerate}


\subsection{Definizione formale di stabilità di un punto di equilibrio}

Preso un punto di equilibrio $x_e$ , questo è stabile
\textbf{semplicemente} se:
comunque scegliamo un intorno di questo punto $(∀ε > 0)$, esiste sempre un intorno
(più piccolo) di quel punto
$(∃δ(ε) > 0)$ tale che, se scegliamo un punto $x_o $ in questo intorno
$(‖x_o - x_e ‖ \le \delta(\varepsilon))$, si avrà che l'evoluzione del sistema nel tempo rimarrà
sempre $(∀t ≥ 0)$ nell'intorno di raggio $ε (‖x(t) - x_e ‖ ≤ ε)$.
\[
\forall \varepsilon>0 \exists \delta(\varepsilon) : ||x_0-x_e||<\delta(\varepsilon) \Longrightarrow ||x(t)-x_e||\le \varepsilon \forall t\ge 0    
\]
È invece stabile \textbf{asintoticamente} se è stabile semplicemente e inoltre al passare del tempo la distanza dei due punti diminuisce:
\[
\forall \varepsilon>0 \exists \delta(\varepsilon)>0 : ||x_0-x_e|| \le \delta(\varepsilon)\Longrightarrow ||x(t)-x_e|| \le \varepsilon \forall t\ge0 \and
\lim_{t\to\infty}||x(t)-x_e|| = 0 
\]


\subsection{Definizione, condizioni e criteri di stabilità interna per sistemi lineari:}
Un sistema è stabile se reagisce in modo positivo all'effetto delle perturbazioni,
nel senso di contenere l'effetto dell'evoluzione libera. 

Per un sistema lineare $\overset{\cdot}{x}(t)=Ax(t)+B u(t)$,
gli stati di equilibrio sono quelli che soddisfano
l'equazione: $\overset{\cdot}{x} = A x_e = 0$, sicuramente verificata per $x_e=0$ e quindi l'origine dello spazio di stato è sempre
lo stato di equilibrio.

Esso è unico se e solo se la matrice dinamica è invertibile  $(det(A)\neq 0)$.
Le proprietà di una stato di equilibrio sono equivalenti a quelle di ogni altro (per cui si ha la possibilità di limitare lo studio allo stato zero);
come conseguenza si parla di stabilità interna come di una proprietà del sistema e non di un particolare stato di equilibrio.

Le condizioni di stabilità sono:
\begin{itemize}
    \item Un sistema è \textbf{semplicemente stabile} se e solo se, preso un valore k finito si ha che ${|e^{At}|\leq k}$ .
    Questa condizione richiede che tutti gli autovalori abbiano
    parte reale minore o uguale a zero se semplici (molteplicià unitaria), e minore di zero se a molteplicità maggiore.
    \item Un sistema è \textbf{asintoticamente stabile} se e solo se è stabile semplicemente,
    e inoltre \(\displaystyle \lim_{t \to\infty }|e^{At}|=0 \), quindi gli autovalori devono essere tutti strettamente minori di 0.
\end{itemize}

Lo studio della \textbf{STABILITÀ INTERNA} si riduce allo studio della stabilità dell'origine: implica la stabilità di tutti i modi.


\subsection{Relazioni tra stabilità esterna e interna}
\begin{enumerate}
    \item stabilità interna $\Longrightarrow$ stabilità esterna.
    \item stabilità esterna $\Longrightarrow$ stabilità esterna nello stato zero.
    \item stabilità esterna nello stato zero $\Longrightarrow$ stabilità esterna se tutti i modi sono eccitabili
    \item stabilità esterna $\Longrightarrow$
    stabilità interna asintotica se tutti i modi sono eccitabili e osservabili.
    \item stabilità esterna $\Longrightarrow$ 
    stabilità interna semplice se tutti i modi sono e osservabili.
\end{enumerate}


\subsection{Stabilità esterna nello stato zero}
Se è verificata la condizione: "$u \text{ limitato} \Longrightarrow y \text{ limitata}$" allora si parla di stabilità esterna.
Un sistema è stabile nello stato zero se e solo se \(\displaystyle \int_{0}^{t}||W(\tau)||d\tau < k \ \forall t \).
In formule è
\[
    \forall\ M>0  \exists N_M>0 : x_0=0 \wedge  ||u(t)||<M\Longrightarrow||y(t)||<N_M
\]
Ovvero la norma della funzione di trasferimento deve essere sommabile nell' intervallo
considerato. Considerando il punto di vista degli autovalori la condizione può
esprimersi come: \textit{gli autovalori che sono simultaneamente eccitabili ed osservabili
devono essere a parte reale negativa.}


\subsection{Definizione di stabilità esterna e condizioni}
La stabilità esterna è anche detta stbilità ingresso-uscita
ed è una proprietà che indica come un sistema risponde
a fronte di ingressi limitati.

Le condizioni necessarie e sufficienti affinché un sistema sia stabile esternamente sono:
\begin{itemize}
\item  \textbf{Stabilità esterna nello stato zero}:
\(\displaystyle \int_{0}^{t} |W(\tau)d\tau|< K\ \forall\ t\),
quindi gli autovalori che sono simultaneamente eccitabili e osservabili
devono avere parte reale negativa.
\item $|\Psi(t)|<k\ \forall t$,
quindi \textit{gli autovalori osservabili in uscita devono essere a parte
reale strettamente negativa se con molteplicità geometrica maggiore di 1,
o parte reale non positiva se con molteplicità geometrica pari a 1.}
\end{itemize}

\section{Dimostazioni formali}

\subsection{Dato un sistema in forma esplicita, dimostra che il sistema è lineare}
Una funzione di variabili reali $f$ è lineare se, dati due punti qualunque
$x$ e $y$, e due costanti $a$ e $b$ anch'esse arbitrarie, si ha che $f(ax+by)=af(x)+bf(y)$.

La rappresentazione esplicita di un sistema è 
\[ x(t) = \Phi(t)x_0 +\int_{0}^{t} H(t-\tau) u(\tau) d \tau
\]
La funzione di stato è
\[ \Psi \left(t_0-t,x_0,\left.u(t)\right|_{(t_0,t]}\right) \]
S è lineare se $\forall$ coppia $(t,t_0) \in (\mathbb{R}x\mathbb{R})$, $\Psi$ è lineare sul prodotto $XxU$ 
(lo spazio degli stati per gli ingressi):

\
\begin{align*}
    \Psi \left(t_0-t,\alpha x_1+\beta x_2,\alpha\left.u_1(t)\right|_{(t_0,t]}+\beta\left.u_2(t)\right|_{(t_0,t]}\right) =\\
    = \alpha\Psi \left(t_0-t,x_1,\left.u_1(t)\right|_{(t_0,t]}\right)+\beta\Psi \left(t_0-t,x_2,\left.u_2(t)\right|_{(t_0,t]}\right)
\end{align*}
\[
    \Psi \left(t_0-t,x_1,\left.u_2(t)\right|_{(t_0,t]}\right)  = \underbrace{\Psi\left(t-t_0,x_1,0\right)}_{\text{evoluzione libera}}+
    \underbrace{\Psi\left(t-t_0,0,\left.u_2(t)\right|_{(t_0,t]}\right)}_{\text{evoluzione forzata}}
\]

La forma esplicita del mio sistema è 
\[
x(t) = \Phi(\alpha x_1+\beta x_2) + \int_{t_0}^{t} H(t-\tau)(\alpha u_1 +\beta u_2)d\tau    
\]
che verifica banalmente la linearità.


\subsection{\boldmath Dimostrazione che data un'evoluzione da $x(t_0)$ a $x(t)$, vale la stessa da $x(t_1)$ con $<t_0<t_1<t$}
Per farlo prendo due equazioni:
\begin{equation}
    x(t) = \Phi(t-t_0)x(t_0)+\int_{t_0}^{t} H(t-\tau)u(\tau)d\tau
\end{equation}
\begin{equation}
    x(t) = \Phi(t-t_1)x(t_1)+\int_{t_1}^{t} H(t-\tau)u(\tau)d\tau
\end{equation}
(Questa roba si trova uguale su edotm ma scrivere mi aiuta a studiare)
Ora voglio dimostrare che le due si equivalgono

\[ x(t) = \Phi(t-t_1)\underbrace{\left[ \Phi(t_1-t_0)x(t_0)+\int_{t_0}^{t_1} H(t-\tau)u(\tau)d\tau\right]}_{x(t_1)}+\int_{t_1}^{t} H(t-\tau)u(\tau)d\tau \]
\[x(t) = \Phi(t-t_1)\Phi(t_1-t_0)x(t_0)+\Phi(t-t_1)\int_{t_0}^{t_1} H(t-\tau)u(\tau)d\tau+\int_{t_1}^{t} H(t-\tau)u(\tau)d\tau\]
\[ \Phi(t-t_0)x(t_0)= \Phi(t-t_1)\Phi(t-t_0)x(t_0)\Longleftrightarrow\Phi(t-t_0)\Phi(t-t_1)\Phi(t-t_0) \]
\[ \int_{t_0}^{t} H(t-\tau)u(\tau)d\tau = \Phi(t-t_1)\int_{t_0}^{t_1} H(t_1-\tau)u(\tau)d\tau+\int_{t_1}^{t} H(t-\tau)u(\tau)d\tau \] 
A sinistra applico la linearità dell'integrale:
\[ \int_{t_0}^{t_1} H(t_1-\tau)u(\tau)d\tau+\int_{t_1}^{t} H(t-\tau)u(\tau)d\tau
 = \Phi(t-t_1)\int_{t_0}^{t_1} H(t_1-\tau)u(\tau)d\tau+\int_{t_1}^{t} H(t-\tau)u(\tau)d\tau \]
e quindi alla fine:
\[ \int_{t_0}^{t_1} H(t-\tau)u(\tau)d\tau = \int_{t_0}^{t_1}\Phi(t-t_1) H(t_1-\tau)u(\tau)d\tau \]
\[ H(t-\tau) = \Phi(t-t_1) H(t_1-\tau) \]
Riassumento
\[\begin{cases} 
        \Phi(t-t_0) = \Phi(t-t_1)\Phi(t_1-t_0)\\
        H(t-\tau) = \Phi(t-t_1) H(t_1-\tau)
\end{cases}
\]
Ora andiamo a derivare la prima equazione del sistema in forma esplicita (1) per vedere se otteniamo
esattamente la forma implicita di partenza:
\[ \overset{\cdot}{x} = \frac{\partial \Phi(t-t_0)}{\partial t}x(t_0)+\int_{t_0}^{t} \frac{\partial}{\partial t}\left( \Phi(t-t_1) H(t_1-\tau)\right)u(\tau)d\tau \]
\[ \overset{\cdot}{x} = \frac{\partial \Phi(t-t_1)}{\partial t}\left(\Phi(t_1-t_0)x(t_0)+\int_{t_0}^{t}H(t_1-\tau)u(\tau)d\tau\right)+H(0)u(t) \]
Ora prendiamo il limite $t_1\to t$:
\[ \overset{\cdot}{x} = \lim_{t_1\to t}\frac{\partial \Phi(t-t_1)}{\partial t}\left(\Phi(t_1-t_0)x(t_0)+\int_{t_0}^{t_1}H(t_1-\tau)u(\tau)d\tau\right)+H(o)u(t) \]
\[ \overset{\cdot}{x} = \lim_{t_1\to t}\frac{\partial \Phi(t-t_1)}{\partial t}x(t)+H(0)u(t) \]
\[ \overset{\cdot}{x} = \frac{\partial \Phi(t\to0)}{\partial t}x(t)+H(0)u(t) \]

Da cui finalmente posso dire che 
\[ \frac{\partial \Phi(t\to 0)}{\partial t}x(t) =\lim_{t\to 0}\frac{\partial \Phi(t)}{\partial t}x(t) = Ax(t) \]
\[ H(0)u(t) = Bu(t) \Rightarrow H(0) = B \]


























\section{Dimostrazioni di risposte a regime permanente}


\subsection{La risposta a regime permanente: definizione e condizioni di esistenza}
La risposta a regime permanente è la funzione intorno alla quale tende la risposta di un sistema
in presenza di un ingresso persistente e indipendentemente dallo stato iniziale.
Si può scrivere come:
\[\lim_ {t_0 \to -\infty} \int_{t_0}^{t} W(t-\tau)u(\tau) d\tau  \]

Affinché esista deve essere assicurata la sommabilità della funzione integranda,
vale a dire che è necessaria la \textbf{limitatezza delle evoluzioni interne (Stabilità asintotica)}
e quindi tutti gli autovalori del sistema devono avere parte reale negativa.



\subsection{Risposta a regime permanente per sistemi a tempo discreto}
La risposta a regime permanente, ad un assegnato ingresso, è quella funzione del tempo intorno alla quale, indipendentemente dallo stato iniziale,
tende ad arrestarsi la risposta in uscita al crescere del tempo.
Per sistemi tempo discreto si definisce con:
\[Y_r(k)=\sum_{\tau =- \infty}^{k} W(k-\tau)u(\tau) \]
Si assume anche in questo caso che il sistema sia stabile internamente (modulo minore o uguale a uno degli autovalori con ordine geometrico unitario,
strettamente inferiore ad uno gli altri), per evitare che vi siano evoluzioni che crescono
indefinitamente impedendone il funzionamento.

Valgono inoltre le ipotesi che i
modi osservabili siano asintoticamente stabili (cioè che i corrispondenti autovalori siano a modulo inferiore ad uno), per assicurare l'indipendenza dallo stato iniziale, e che
il sistema sia esternamente stabile nello stato zero (modulo inferiore ad uno dei
poli della funzione di trasferimento).


\subsection{Risposta a regime permanente ad ingressi sinusoidali per sistemi a tempo continuo}
La risposta a regime permanente è quell'andamento
nel tempo al quale tende ad assestarsi la risposta
quando l'ingresso è persistente e indipendente dallo stato.
La risposta a regime permanente per ingressi sinusoidali
è una funzione simile all'ingresso, ma modificata in modulo e fase, assume la forma :
\[ Y_r(t)=M(\omega)sen(\omega t + \Phi(\omega))\]
Il modulo è dato da $|W(j\omega)|=\sqrt{Re^2+Im^2}$,
lo sfasamento invece è $ L(W(j\omega))=\Phi(\omega)=arctan \frac{Im}{Re}$.


\subsection{\boldmath Dimostrare la formula della risposta a regime permanente all'ingresso $u(k) = sin(\theta k)$ per un sistema a tempo discreto.}
Si ricorda che per i sistemi tempo discreto

\[Y_r(k)=\sum_{\tau =- \infty}^{k} W(k-\tau)u(\tau)\]e che
$u(k)=\frac {e^{j\theta k} - e^{-j\theta k}}{2j} $.


Considero inizialmente l'ingresso $u(k)=e^{j \theta k}$ e ottengo:
\[ 
    Y_r(k)=\sum_{\tau =- \infty}^{k} W(k-\tau)e^{j\theta \tau},
    \quad \text{ora pongo }\quad k-\tau=\xi
\]
Così viene:
\[ Y_r(k)=\sum_{\xi =0}^{\infty} W(\xi)e^{j\theta (k-\xi)} = e^{j\theta k} \sum_{\xi =0}^{\infty} W(\xi)e^{-j\theta \xi}
= e^{j\theta k} W(Z)|_{e^{j \theta}} \].
Di conseguenza la risposta a regime permanente a questo ingresso è:

$Y_r(k)=e^{j \theta k}W(e^{j \theta})$.
Considero ora l'ingresso $u(k)=e^{-j \theta k}$, seguendo calcoli molto simili al caso precedente,
e grazie a questi due risultati ottengo la risposta
nel caso di ingresso sinusoidale:
\[
    Y_r(k)=\frac{e^{j \theta k}W(e^{j \theta})-e^{-j \theta k}W(e^{j \theta})}{2j}
    = M(\theta)\sin(\theta t +\phi(\theta))
\]


\subsection{\boldmath Si dimostri che, se esiste, la risposta a regime permanente a $u(t) = \delta _{-1}(t)$(ingresso a gradino) è uguale al guadagno del sistema}
Considerando l'ingresso $u(t) = \delta_{-1}(t)$ un ingresso costante,
la risposta a regime sarà del tipo 
\[ 
    Y_r(t)= \lim_{t_0 \to -\infty} \int_{t_0}^{t}W(t-\tau)u(\tau)d\tau=\int_{-\infty}^{t}W(t-\tau)\cdot 1 d\tau
\]
Applicando il cambio di variabile $t-\tau = \theta$ ottengo $Y_r(t)=\int_{0}^{+\infty}W(\theta) d\theta$.

Questo integrale assomiglia al calcolo della trasformata di Laplace $W(s)$;
manca però il termine $e^{-s\theta}$, che tuttavia è nullo per $s=0$.
Quindi posso moltiplicare l'integrale per  $e^{-s\theta}$ e poi calcolarlo per $s=0$:
\[ 
    Y_r(t)=\int_{0}^{+\infty}W(\theta) d\theta =\left. \int_{0}^{+\infty}W(\theta) e^{-s\theta} d\theta \right|_{s=0} =\left. W(s)\right|_{s=0}
\]
Come sappiamo $W(0)$ mi restituisce il guadagno della funzione perciò la risposta a regime permanente dell'ingresso costante è pari al guadagno.

\subsection{Dimostrare che la funzione di trasferimento W(s) non dipende dalla scelta delle coordinate}
Sappiamo che $W(t) = Ce^{At}B+D$\\
Se effettuo il cambio di coordinate 
\[ z=Tx \quad \to \quad \widetilde{A}=TAT^{-1}, \widetilde{B}=TB, \widetilde{C}=CT^{-1}, \widetilde{D}=D\]
ottengo:
$\widetilde{W}=\widetilde{C}e^{At}\widetilde{B}+\widetilde{D}=CT^{-1}Te^{At}T^{-1}TB+D=W(t)$

\subsection{È possibile che un sistema abbia funzione di trasferimento uguale a zero? Se si, sotto
quali condizioni?}
Si, è possibile perché essendo 
\[ W(s)=\mathscr{L}[W(t)] \quad W(t)=Ce^{At}B=\sum_{i=1}^{n}e^{\lambda i}Cu_iv_iB \]
affinché un modo non compaia nella $W(t)$ deve essere non eccitabile oppure non osservabile.






\section{Esercizi}

\subsection{Trovare il valore massimo del parametro r tale per cui il sistema retroazionato sia stabile. }

\incfig{sistema.jpeg}{1.2}

Dato che mi sfugge una maniera miglioe di farlo lo faccio provando gli indici. Ovviamente si inizia da $r=0$:
\[ 
    W_{AP} = \frac{k(s-1)}{(s+1)} \to W_{CH} = \frac{k(s-1)}{(k+1)s+1-k}
\]
Con Routh vediamo il caso $k+1>0$:
\begin{align*}
    1+k >0 &\to& k>-1&\to& k >0\\
    1-k >0 &\to& k<1
\end{align*}
e $k+1 < 0\to k <-1$:
\begin{align*}
    k+1 <0 &\to& k < -1\\
    1-k <0 &\to& 1<k 
\end{align*} 
Che non viene.
Quindi vabbe per $0<k<1$ è stabile asintoticamente.

Ora andiamo avanti e proviamo 
\[ 
    W_{AP} = \frac{k(s-1)}{s(s+1)} \to W_{CH} = \frac{k(s-1)}{s^2+(k+1)s-k}
\]
e sempre con Routh viene:
$K>0$ sicuro no chè $s^2$ ha coefficiente positivo $(1)$, con $k<0$ invece:
\begin{align*}
    1>0\\
    k+1>0 &\to& k>-1\\
    -k>0 &\to& k <0
\end{align*}
Quindi è stabile per $-1<k<0$
Ora andiamo avanti e proviamo 
\[ 
    W_{AP} = \frac{k(s-1)}{s^2(s+1)} \to W_{CH} = \frac{k(s-1)}{s^3+s^2+ks-k}
\]
Che viene 
\begin{align*}
    3\text{ grado}&1&k\\
    2\text{ grado}&1&-k\\
    1\text{ grado}& 2k= \frac{\begin{vmatrix}
    1 & k \\
    1 & -k
    \end{vmatrix}}{-1} \\
    0\text{ grado}&-k
\end{align*}
Che avendo sia $2k$ che $-k$ non può venire mai bene.

\subsection{Trovare $P_2$ }

\incfig{p1ps.jpeg}{0.5}

\[ L[u] = \frac{1}{s} \quad,\quad L[y] = \frac{1}{s-1}\]
\[ y(s) = u(s)W(s) \quad\to\quad W(s)=\frac{1}{s+1}P_2(s) \]
\[ \frac{1}{s-1} = \frac{1}{s}\frac{1}{s+1}P_2(s) \]
\[ \frac{s(s+1)}{s-1} = P_2(s) \to W_{tot}=\frac{s}{s-1} \]



\end{document}
